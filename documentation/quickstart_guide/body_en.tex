\begin{center}
\huge QuickStart Guide\\[5.5mm]
\end{center}
\large{This guide provides an example of setting up an openvpn network in 7 steps 
using the \Nbm{}}.\\[1mm]

\colhead{Start}
\picContentRight{config}{
First, unzip the \Nbm{} and start it by clicking on the
"start-manager"-file in the unzipped directory. When the \Nbm{} runs for the
first time, it will guide you trough a basic configuration.
}{width=85mm,height=60mm}

\colheadRight{Create the users}
\picContentLeft{user1}{
After the basic configuration you can now work with the \Nbm{}. The next step
is to create all neccessary user. Click on ``Edit-$>$Create new user'' and a
UserCreation-Dialog will appear.
}{width=75mm,height=52mm}

\colhead{Create the server}
\picContentRight{server1}{
Click on ``Edit-$>$Create new server'' to create your server. If you have a
boa-server, you can select this in the first tab under server-type. This will
save you time configuring the server.\\
}{width=76mm,height=56mm}

\newpage

\colheadRight{Syncing}
\picContentLeft{sync}{
By syncing the \Nbm{} pushes all files which are relevant for the server, to
it.
}{width=76mm,height=56mm}

\colhead{Setup the server}
\picContentRight{startVpn}{
The \Nbm{} will automatically create a OpenVPN-Server-Config. You only need
to push it to the server as well as start the openvpn-daemon on the
server.\\
If you have a boa, the \Nbm{} will push the config for you in the syncing
process.
}{width=76mm,height=56mm}

\colheadRight{Connect and Test}
\picContentLeft{control}{
One of the features of the \Nbm{} is that it automatically creates
userconfig-files. Simply switch into the export-dir and start the
openvpn application.\\
Another cool feature is the controll tab of the \Nbm{}.
You can enable it in
the configuration tab. Then select the server (not open it) and click on the
Control Center-Button in the main Button-Panel.\\
Keep in mind that you can only use the Control Tab if an openvpn process is
running on the server. In the tab you can see all connected clients or you
can send your own commands to the server.
}{width=80mm,height=56mm}

\colhead{Where to go from here}
This quickstart guide is only capable of giving a brief introduction into
the \Nbm{}. Therefore it left out background information or some cool
features of the \Nbm{}.
The good news is that you can explore all this in the \Nbm{}-Handbook.
